\section{Motivation}
\label{sec:motivation}

Suppose someone implemented a clever algorithm for
regular expression matching. 
We want to use this implementation, but we also want to play safe and
make sure it is largely bug free by subjecting it to extensive
testing---verification is deemed to expensive.
Such testing requires us to come up with test cases and
implement a test oracle.


A test case consists of a regular expression \code{r} and an input
string \code{s}. If \code{match} is the implementation of matching
and \code{matchOracle} is the test oracle, then
executing the test case means to execute \code{match r
  s} and check whether the result is correct by comparing it with
\code{matchOracle r s}. 

A popular way of conducting such a test is using the QuickCheck library
\cite{quickcheck}, which performs property-based random testing. Using
QuickCheck, we would write a random generator for regular expressions
and then use the random generator for strings to generate many inputs for a
generated regular expression.

However, this approach has a
catch. Depending on the language of the regular expression, the
probability that a random string is a member of the language can be
severely skewed. As an example, consider the language $L = (ab)^*$ over the
alphabet $\Sigma = \{a, b\}$. Although $L$ contains infinitely many
words, the probability that a random word of
length $n$ is an element of $L$ is
\begin{itemize}
\item $0$ if $n$ is odd and
\item $\frac{1}{2^n}$ if $n$ is even.
\end{itemize}
Thus, the probability $p_n$ that a random word of length less than or equal to
$n$ is an element of $L$ is very small:
\begin{align*}
  p_n &= \frac{\lfloor n/2 \rfloor}{2^{n+1} - 1}
        \le \frac{n}{2^{n+2} - 2}
\end{align*}
Hence, the probability of (uniformly) randomly
selecting a word in $L$ is zero in the limit.

Hence, there are two problems with testing the regular expression
matcher.
\begin{enumerate}
\item How do we know whether the test oracle is correct, short of
  verifying it?
\item How do we ensure that relevant test cases are generated, given
  that the probability of randomly picking a word in the language is
  $0$ or $1$ for many regular languages?\footnote{Exercise for the
    interested reader:
    \begin{enumerate}
    \item Come up with a regular language $R$ such that $P(w\in R)$ is
      different from $0$ or $1$.
    \item Given a proper fraction $m/n$ (that is $n>0$ and $0\le m\le
      n$) define a regular language $R$ such that $P (w\in R) = m/n$.
    \end{enumerate}
  }
\end{enumerate}


Wouldn't it be nice to have a systematic and obviously correct means
of generating words \textbf{inside} of $L$ and \textbf{outside} of
$L$? Such a generation algorithm would obviate the need for an oracle
and it would make sure that we can control the number of test inputs
in the language and in the language's complement.

\subsection{Research Question}
\label{sec:research-question}

\begin{figure}[tp]
  \begin{align*}
    r, s & &L(\_)=\quad &  &
                             \makebox[1em][l]{\code{data GRE sig}}\\
         & ::= \Rnull & \ltext{empty}
                        & \emptyset
                           &&\code{= Zero}\\
         & \mid \Rempty & \ltext{empty word}
                        & \{ \varepsilon \}
                           && \code{| One}\\
         & \mid (a \in \Sigma) & \ltext{singleton}
                        &  \{ a \}
                           && \code{| Atom sig} \\
         & \mid \Runion rs & \ltext{alternative}
                        &  L (r) \cup L (s)
                           && \code{| Or (GRE sig) (GRE sig)}\\
         & \mid \Rconcat rs & \ltext{concatenation}
                        &  L (r) \cdot L (s)
                           && \code{| Dot (GRE sig) (GRE sig)}\\
         & \mid \Rstar r & \ltext{Kleene star}
                        & (L (r))^* 
                           && \code{| Star (GRE sig)}\\
         & \mid \Rintersect rs & \ltext{intersection}
                        & L (r) \cap L (s)
                           && \code{| And (GRE sig) (GRE sig)}\\
         & \mid \Rcomplement r & \ltext{complement}
                        & \Sigma^* \setminus L (r)
                           && \code{| Not (GRE sig)}
  \end{align*}
  % \begin{align*}
  %   L (\Rnull) &= \emptyset\\
  %   L (\Rempty) &= \{ \varepsilon \} \\
  %   L (a) &= \{ a \} \\
  %   L (\Runion rs) &= L (r) \cup L (s) \\
  %   L (\Rconcat rs) &= L (r) \cdot L (s) \\
  %   L (\Rstar r) &= (L (r))^* \\
  %   L (\Rintersect rs) &= L (r) \cap L (s) \\
  %   L (\Rcomplement r) &= \Sigma^* \setminus L (r)
  % \end{align*}
  \caption{Generalized regular expressions}
  \label{fig:generalized-regular-expressions}
\end{figure}

In the follwing we will tackle a slightly more general question, namely generating
the language of a \emph{generalized regular expression}, 
which subsumes the purpose of generating positive and negative sample
words for testing.

A generalized regular expression
(Figure~\ref{fig:generalized-regular-expressions}) is an expression
built from the regular operators empty set, empty word, singleton word
consisting of a single letter $a$ chosen from a finite alphabet
$\Sigma$, alternative, concatenation, and Kleene star. In addition, it
may contain the operators intersection and complement.

As customary, we write $\Sigma^*$ for the set of finite words over
alphabet $\Sigma$, which is defined inductively as the smallest set
such that $\varepsilon \in
\Sigma^*$ and $\forall a\in\Sigma, \forall w\in\Sigma^*, aw \in \Sigma^*$.
The semantics of an expression, $L(r) \subseteq \Sigma^*$, is a set of
words, which is also defined in
Figure~\ref{fig:generalized-regular-expressions}. It relies on
standard definitions from the theory of formal languages. We write
$\varepsilon$ for the empty word and $u\cdot v$ for the concatenation
of words $u, v \in \Sigma^*$. If $U, V \subseteq \Sigma^*$ are
languages, then their concatenation (or product) is defined as $U\cdot
V = \{ u\cdot v \mid u\in U, v\in V\}$. The Kleene closure of a
language $U\subseteq \Sigma^*$ is defined as $U^* =
\bigcup_{i=0}^\infty U^i$ where $U^0 = \{\varepsilon\}$ and $U^{i+i} =
U \cdot U^i$. 

\subsection{Naive Approach}
\label{sec:naive-approach}

We start with a naive implementation of the mathematical definition in
Figure~\ref{fig:generalized-regular-expressions}. We define an
alphabet by a list of \code{Char}.  We represent words by elements of Haskell's
\code{Data.Text.Text} datatype, abbreviated to \code{T.Text}. We
represent a language as a lazy list of \code{Text}, as a language
can be an infinite set. There are two further restrictions.
\begin{enumerate}
\item The output of a generator should not contain repetitions
  because we would like to guarantee that test inputs are different
  from each others.
\item The output of a generator should not be partial because it would
  lead the test code to hang on a nonterminating input.
\end{enumerate}

\begin{lstlisting}
import Data.Text as T

type Alphabet = [Char]
type Lang = T.Text

generate :: Alphabet -> GRE Char -> Lang
generate sigma r = gen r
  where
    gen Zero = []
    gen One  = [T.empty]
    gen (Atom t) = [T.singleton t]
    gen (Or r s) = union (gen r) (gen s)
    gen (Dot r s) = concatenate (gen r) (gen s)
    gen (Star r) = star (gen r)
    gen (And r s) = intersect (gen r) (gen s)
    gen (Not r) = complement sigma (gen r)
\end{lstlisting}
\begin{figure}[tp]
\begin{verbatim}
module Examples.Naive where

union :: Lang -> Lang -> Lang
union lx ly = lx ++ ly

concatenate :: Lang -> Lang -> Lang
concatenate lx ly = [T.append wx wy | wx <- lx, wy <- ly ]

intersect :: Lang -> Lang -> Lang
intersect lx ly = [wx | wx <- lx, wx `elem` ly ]

star :: Lang -> Lang
star lx = concat lxi
  where
    lxi = [T.empty] : map (concatenate lx) lxi

complement :: Alphabet -> Lang -> Lang
complement sigma lx =
  undefined
\end{verbatim}
  \caption{Partial implementation of the regular operators}
  \label{fig:regular-operators-0}
\end{figure}
As a basis for our further discussion, we exhibit a partial 
implementation in Figure~\ref{fig:regular-operators-0}.
This implementation has a number of deficiencies.
\begin{description}
\item[union] The output may contain duplicates. More seriously, if
  \texttt{lx} is infinite, then no words from \texttt{ly} will ever be
  produced. This behavior violates the specification of set union
  because there may be elements in \texttt{ly} that never appear in
  \texttt{union lx ly}.

  If we restricted ourselves to finite lists, then replacing
  \texttt{++} with \texttt{Data.List.union} would be an appropriate
  implementation, but its worst-case complexity is quadratic.
\item[concatenate] The output may contain duplicates. If \texttt{ly}
  is infinite, then only the first word in \texttt{lx} contributes to
  the output.

  For finite lists, an appropriate implementation would compose the
  raw product computation with \texttt{Data.List.nub} to remove
  duplicates. The worst-case complexity of \texttt{nub} is quadratic.
\item[intersect] The output contains no duplicates, if \texttt{lx}
  does not contain duplicates, either. If \texttt{ly} is infinite,
  then the resulting list may be partial because the \texttt{elem}
  operation may not terminate.

  For finite lists, this implementation would be appropriate.
\item[star] The output may contain duplicates. If \texttt{lx} is
  infinite, then the generated language is just \texttt{T.empty : lx}.

  If \texttt{lx} was guaranteed to be finite, then \texttt{star} can
  be implemented in a way that guarantees no duplication. However, to
  retain finiteness, we would have to impose an arbitrary limit on the
  size of the output.
\item[complement] In general there is no computable way to determine
  whether a word occurs in a lazy list \texttt{lx}. Hence, we have no
  good definition to propose.

  If \texttt{lx} was guaranteed to be finite, then it is possible to
  enumerate its complement without repetitions. Again, to retain
  finiteness, we would have to impose an arbitrary limit on the size
  of the output.
\end{description}
\begin{figure}[tp]
\begin{verbatim}
module Examples.Finite where

import Data.List as L

limit :: Int
limit = 1024

union :: Lang -> Lang -> Lang
union lx ly = L.union lx ly

concatenate :: Lang -> Lang -> Lang
concatenate lx ly = L.nub [T.append wx wy | wx <- lx, wy <- ly ]

intersect :: Lang -> Lang -> Lang
intersect lx ly = [wx | wx <- lx, wx `elem` ly ]

star :: Lang -> Lang
star lx = take limit $ removeDuplicates $ concat lxs
  where
    lxs = [T.empty] : map (concatenate lx1) lxs
    lx1 = L.delete T.empty lx
    removeDuplicates [] = []
    removeDuplicates (w:ws) = w : removeDuplicates (filter (!=w) ws)

complement :: Alphabet -> Lang -> Lang
complement sigma lx = take limit (concat lsigmastar L.\\ lx) 
  where
    lsigmastar =
      [T.empty] : 
      map (\lsigmai -> concatMap (\la -> concatenate la lsigmai) lsigma1) lsigmastar
    lsigma1 = map T.singleton sigma
\end{verbatim}
  \caption{Finite implementation of the regular operators}
  \label{fig:finite-regular-operators}
\end{figure}
Figure~\ref{fig:finite-regular-operators} contains an implementation
of a finite version of the generator module. The implementation of
\texttt{star} follows the definition of $U^*$ literally. It first
recursively creates a list \texttt{lxs} of the iterates $U_\bullet^i$ where
$U_\bullet = U \setminus \{\varepsilon\}$, concatenates all of
them\footnote{It is easy to see that $U^* = U_\bullet^*$.}, removes
duplicates, and imposes the limit. Removing duplicates introduces quadratic
worst-case time complexity in the size of the output.

The implementation of \texttt{complement} generates the language
$\Sigma^*$ analogously to the construction of $U^*$ in
\texttt{star}, uses the list difference operator
\texttt{L.\textbackslash\textbackslash} to remove elements of
\texttt{lx}, and finally imposes the limit. Its worst-case run time is
$O(m\cdot n)$ where $m$ is the limit and $n = |\texttt{lx}|$.

In summary, the naive approach in
Figure~\ref{fig:regular-operators-0} can generate infinite languages,
but has many drawbacks that lead to duplication and incompleteness
(words in the language are not enumerated). Moreover, the complement
is not computable for this approach.

The finite approach in Figure~\ref{fig:finite-regular-operators}
imposes an arbitrary limit on the number of generated words. This
limit can lead to omitting words in nonempty languages where $P (w\in
R) = 0$. Moreover, there are many places (in \texttt{union},
\texttt{concatenate}, and \texttt{star}) with quadratic worst-case
time complexity. 

At this point, the question is: Can we do better? Can we come up with
a generator that supports finite as well as infinite languages
efficiently without incurring extraneous quadratic behavior?


\subsection{Ordered Enumeration}
\label{sec:ordered-enumeration}


\subsubsection{Union}
Ideally, the \code{union} operator it should behave like
set union:
\begin{lstlisting}
w `elem` union xs ys   <==>   w `elem` xs || w `elem` ys
\end{lstlisting}
Can we implement \code{union} simply by list append?
No, we cannot. 
If \code{xs} is an infinite list and \code{ys} is different from
\code{xs}, then there is some \code{w `elem` ys} such that
\code{w `elem` union xs ys} does not terminate.
If \code{xs} is a finite list without duplicates, then simply
appending \code{ys} may result in duplicates.
Another alternative, \code{Data.List.union}, would apply only to finite lists.

Can we implement \code{union} by interleaving? No, we cannot for
similar reasons.

Our approach is to generate a language as a potentially infinite list
in ascending \emph{length-lexicographic order}. The
length-lexicographic ordering is defined by $u \lleq 
v$ if $|u|<|v|$ or $|u|=|v|$ and $u\le v$ in the usual lexicographic
ordering. Here is a definition in Haskell.
\begin{lstlisting}
lleq :: T.Text -> T.Text -> Bool
lleq u v =
  let lxs = T.length xs
      lys = T.length ys
  in  lxs < lys ||
      lxs == lys && xs <= ys
\end{lstlisting}
This ordering has the advantage that it gives raise to a standard
enumeration of all words over a totally ordered alphabet. In
particular, for each pair of words $u \lleq v$ there is only a finite
number of words $w$ such that $u \lleq w$ and $w \lleq v$. This property does
not hold for the lexicographic ordering where all elements of $a\cdot
(a+b)^*$ lie between words $a$ and $b$ (assuming $\Sigma = \{a, b\}$
with $a<b$).

With the LLO representation, the \code{union} operation can be
implemented by merging two lazy lists.
\begin{lstlisting}
union :: Lang -> Lang -> Lang
union [] ys = ys
union xs [] = xs
union xs@(x:xs') ys@(y:ys')
  | x == y   = x : union xs' ys'
  | lleq x y = x : union xs' ys
  | otherwise  = y : union xs ys'
\end{lstlisting}
Moreover, this operation runs in linear time. If the input languages
are finite of size $m$ and $n$, respectively, then $O(m+n)$
comparisons, pattern matches, and cons operations are needed.

This implementation fulfills the specification to a much larger
degree, but it still gets stuck on partial lists.

%%% Local Variables:
%%% mode: latex
%%% TeX-master: "main"
%%% End:
