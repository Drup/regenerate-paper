\section{Testing}
\label{sec:test}

We implemented our algorithms as libraries to create
test harnesses for regular expression implementations.
We used this library to implement a test harness for the \ocaml \code{Re} library
\footnote{\url{https://github.com/ocaml/ocaml-re}},
which is one of the most used \ocaml regular expression implementation.
%
We also used created a set of test cases for students projects in \haskell
which helped them write better implementations.

Concretely, the library provides a test-harness which generate
regular expressions along with both positive and negative samples. The
implementation under test can then compile the regular expression and apply it
efficiently on the samples.
The library exposes the sample generation as a generator in the style of
property testing such as QuickCheck~\cite{DBLP:conf/icfp/ClaessenH00}.
This allows to use all the tooling already available in such libraries, for
example to generate arbitrary regular expressions.
%
The simplified API of the \ocaml version is shown below.
The main function \code{arbitrary n alphabet} returns a generator
which provides on average \code{n} samples using the given alphabet.

\begin{lstlisting}
type test = {
  re : Regex.t ;
  pos : Word.t list ;
  neg : Word.t list ;
}
val arbitrary:
  int -> Word.char list -> test QCheck.arbitrary
\end{lstlisting}

Regular expressions are represented as an algebraic datatypes which
are easy to generate using QuickCheck-like libraries.
The only constraints we place on the generated regular expressions
is to restrict the star-height to less than 3. While our technique can be used
for regular expressions with several nested repetitions, it can cause
occasional slowdown and large memory consumption which are inconvenient
in the context of automated testings.

Our testing library only returns a finite number of samples. However, the
language can (and often will) be infinite. We want to generate test-cases that will,
on average, exercise the most the implementation to test. For this purpose, we
use a technique similar to the fast approximation
for reservoir sampling~\citep{DBLP:journals/toms/Vitter87}.
When considering
the sequence of words in the language, we skip $k$ elements where
$k$ follows a power law of mean $n$. We then return the given
sample, and stop the sampling with a probability $1/n$.
%
Using this technique, we obtain on average $k$ samples that are regularly
spaced out in the front of the stream, but will occasionally skip ahead
and return very large words. This has proven satisfying at finding good
testing samples in practice.

%%% Local Variables:
%%% mode: latex
%%% TeX-master: "main"
%%% End:
