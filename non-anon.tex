
\documentclass{acmart}
\usepackage{hyperref}
\title{Regenerate: Generation of test cases for regular expression engines}
\subtitle{Supplemental material (non-anonymous)}
\settopmatter{printfolios=false,printccs=false,printacmref=false}

\begin{document}

\maketitle

We implemented our algorithm for test-cases generation in OCaml and Haskell.
We also made a Web application based on the OCaml version, which
is available on \url{https://regex-generate.github.io/regenerate/}.
Both implementations are open source, and can be found at the following
addresses:
\begin{itemize}
\item OCaml: \url{https://github.com/regex-generate/regenerate}\\
  It can be installed directly with opam: \texttt{opam install regenerate}.
\item Haskell: \url{https://github.com/peterthiemann/re-generate}
\end{itemize}

We also implemented a test-harness for the OCaml \texttt{Re} library.
The test harness is available in the OCaml sources in the \href{https://github.com/regex-generate/regenerate/tree/master/test/re}{\texttt{test/re/}} directory. You can run it by installing all the required dependencies (\texttt{opam install -t --deps regenerate}) and running the tests (\texttt{make test}).
  

\end{document}


%%% Local Variables:
%%% mode: latex
%%% TeX-master: t
%%% End:
