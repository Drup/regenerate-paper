\section{Introduction}

Regular languages are everywhere. Due to their apparent simplicity and
their concise representability in the form of regular expressions,
regular languages are used for many text processing
applications, reaching from text editors
\cite{DBLP:journals/cacm/Thompson68} to extracting data from web
pages.

Consequently, there a many algorithms and libraries that implement
parsing for regular expressions. Some of them are based on Thompson's
translation from regular expressions to nondeterministic finite
automata and then apply the powerset construction to obtain a
deterministic automaton. Others are based on Brzozowski's derivatives
\cite{Brzozowski1964} and
map a regular expression directly to a deterministic
automaton. Antimirov's partial derivatives \cite{Antimirov96Partial}
yield another transformation into a nondeterministic automaton. An
implementation based on Glushkov automata has been proposed
\cite{DBLP:conf/icfp/FischerHW10} with decent performance.
Cox's webpage \cite{cox07:_implem_regul_expres} gives a good overview
of efficient implementations of regular expression search. It includes
a discussion of his implementation of Google's RE2 \cite{cox10:_regul_expres_match_wild}.
Current research still uncovers new efficient algorithms for matching
subclasses of regular expressions \cite{DBLP:journals/jcss/GrozM17}.

Some of the algorithms for regular expression matching are rather
intricate and the natural question arises how to test these algorithms. 
While there online repositories with reams of real life regular
expressions \cite{regul_expres_librar}, there are no satisfactory
generators for test inputs. It is not too hard to come up with
generators for strings that match a given regular expression, but that
is only one side of the medal. On the other hand, the algorithm should
reject strings that do not match the regular expression, so it is
equally important to come up with strings that do \textbf{not} match.

This work presents generator algorithms for extended regular expressions that
contain intersection and complement beyond the regular operators. The
presence of the complement operator enables the algorithms to generate
strings that certainly do not match a given (extended) regular
expression.

Our implementations are useful in practice. They are guaranteed to be
productive and produce total outputs. That is, a user can gauge the
string size as well as the number of generated strings without risking
partiality.

Even though the implementations
are not tuned for efficiency they generate
languages at a rate between $16\cdot10^3$ and $1.4\cdot10^6$ strings per
second, for Haskell, and up to $3.6\cdot10^6$ strings per second, for
OCaml. The generation rate depends on the density of the language.


The development of the generator is explained using both Haskell and
OCaml and implementations for both languages will be available for
artifact evaluation. The overall design makes significant use of
laziness. This part of the design space is investigated and explained
using Haskell in Sections~\ref{sec:motivation}
and~\ref{sec:improvements}. Fine tuning of the underlying data
structures is investigated using OCaml in Section~\ref{sec:ocaml}.
Section~\ref{sec:bench} reports benchmarking results,
Section~\ref{sec:related-work} discusses related work, and
Section~\ref{sec:conclusions} concludes. 

We assume fluency with Haskell and OCaml throughout the paper.  Some
familiarity with formal languages is helpful, but not required as the
paper contains all relevant definitions. Our notation for formal
languages is borrowed from one of the classic textbooks on the topic
\cite{DBLP:books/daglib/0011126}.

%%% Local Variables:
%%% mode: latex
%%% TeX-master: "main"
%%% End:
