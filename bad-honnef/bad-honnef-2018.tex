\documentclass{article}
% Encoding and lang
\usepackage[T1]{fontenc}
\usepackage[utf8]{inputenc}
\usepackage[english]{babel}

% Graphical packages
\usepackage{graphicx}
\usepackage{xcolor}

% Math
\usepackage{amsmath}
\usepackage{amsfonts}
\usepackage{amssymb}
\usepackage{amsthm}
% \usepackage{mathrsfs}
\usepackage{mathtools}
% \usepackage{textcomp}
% \usepackage{textgreek}
\usepackage{cmll}
\usepackage{mathpartir}
\usepackage{url}

% Good citations and bibliography
\usepackage{natbib}

% % Pictures and latex lists
% \usepackage{wrapfig}
% \usepackage{float}
\usepackage{caption}
\usepackage{subcaption}
% \usepackage{placeins}
% \usepackage{array}
% \usepackage{paralist}
\usepackage{enumitem}% Compact lists
\usepackage{pifont}

% Specialized packages
% \usepackage{syntax} % Grammar definitions
\usepackage{verbatim}
\usepackage{listings} % Code
\usepackage{xspace} % Useful for macros
\usepackage{mathpartir}

\usepackage{textcomp}

\usepackage[noabbrev,nameinlink,capitalize]{cleveref}

% Custom macros

\newcommand{\ie}[0]{{i.e.}, }
\newcommand{\eg}[0]{{e.g.}, }

\newcommand\TODO[1]{{\ \\\color{red}\large\textbf{TODO} #1}\\}

\newcommand\mysc[1]{{\textsc{#1}}\xspace}
\newcommand\ocaml{\mysc{OCaml}}
\newcommand\haskell{\mysc{Haskell}}

\newcommand\htag[1]{\shortintertext{\textbf{#1}}}


\newcommand\code[2][]{\mbox{\lstinline[#1,basicstyle=\ttfamily\normalsize]{#2}}}

% Colors!
\definecolor{butter}{HTML}{C4A000}
\definecolor{orange}{HTML}{CE5C00}
\definecolor{chocolate}{HTML}{8F5902}
\definecolor{chameleon}{HTML}{4E9A06}
\definecolor{skyblue}{HTML}{204A87}
\definecolor{plum}{HTML}{5C3566}
\definecolor{scarletred}{HTML}{A40000}
\definecolor{lightalu}{HTML}{BABDB6}
\definecolor{darkalu}{HTML}{2E3436}
\newcommand{\kwstyle}{}

\lstdefinelanguage{Haskell}{
  otherkeywords={|,=>,<=,<,>,::,=,@,||,\$},%
  keywords=[1]{if,then,else,case,in,let,where,do},
  keywords=[2]{class,data,newtype,of,deriving,type,sig},%
  keywords=[3]{hiding,infix,infixl,infixr,import,instance,module,qualified},%
  keywordstyle=\kwstyle,
  keywordstyle=[1]\kwstyle\color{chameleon},
  keywordstyle=[2]\kwstyle\color{scarletred},
  keywordstyle=[3]\kwstyle\color{skyblue},
  keywordstyle=[4]\kwstyle\color{butter},
  keywordstyle=[5]\kwstyle\color{skyblue},
  keywordstyle=[6]\kwstyle\color{skyblue},
  keywordstyle=[7]\kwstyle\color{chameleon},
  keywordstyle=[8]\kwstyle\color{butter},
  keywordstyle=[9]\kwstyle\color{butter},
  sensitive,%
  comment=[l]{--},%
  comment=[n]{\{-}{-\}},%
  string=[b]",%
  literate={->}{{{\kwstyle\color{chameleon}->}}}2
}%

%% Code listing
\lstset{
  tabsize=4,
  aboveskip={0.5\baselineskip},
  belowcaptionskip=0.5\baselineskip,
  columns=fixed,
  showstringspaces=false,
  extendedchars=true,
  breaklines=true,
  frame=none,
  basicstyle=\small\ttfamily, %\scriptsize\ttfamily
  keywordstyle=\bfseries,
  commentstyle=\itshape\color{gray},
  % identifierstyle=\color{blue!80!black},
  stringstyle=\color{purple!40!black},
  numbersep=5pt,
  numberstyle=\tiny\color{gray},
  escapeinside={(*@}{@*)},
  numbers=left,
  emphstyle=\color{green!60!black}\bfseries,
  emphstyle={[2]\color{blue!60!black}\bfseries},
  language=Haskell
}




%%% regular expressions
\newcommand\Rnull{\mathbf0}
\newcommand\Rempty{\mathbf1}
\newcommand\Runion[1]{#1 +}
\newcommand\Rconcat[1]{#1 \cdot}
\newcommand\Rstar[1]{#1^*}
\newcommand\Rintersect[1]{#1 \with}
\newcommand\Rcomplement{\texttildelow}

%%% misc
\newcommand\ltext[1]{\makebox[0pt][r]{\text{#1}\quad}}
\newcommand\lleq{\le_{ll}}

%%% Local Variables:
%%% mode: latex
%%% TeX-master: "main"
%%% End:


% Declare a new environment for floating examples. Used like figure.
\usepackage{newfloat}
\DeclareFloatingEnvironment[
listname={List of Examples},
name={Example},
placement={!ht},
]{ex}
\crefname{ex}{Example}{Examples}
\Crefname{ex}{Example}{Examples}
\crefname{subex}{Example}{Examples}
\Crefname{subex}{Example}{Examples}

% Bibliography
% \bibliographystyle{ACM-Reference-Format}
% \citestyle{acmauthoryear}
\bibliographystyle{plain}

\begin{document}
\title{Generating Tests for Regular Expression Engines}

\author{Gabriel Radanne \qquad Peter Thiemann \\ University of Freiburg, Germany}
% \email{radanne@informatik.uni-freiburg.de}


% \author{Peter Thiemann}
% \affiliation{
%   \institution{University of Freiburg}
%   \country{Germany}
% }
% \email{thiemann@acm.org}

\maketitle

\begin{abstract}
  Regular expressions are part of every programmer's toolbox.  They are
used for a wide variety of languge-related tasks and there are many algorithms for
manipulating them. In particular, matching algorithms that detect
whether a word belong to the language described by a regular
expression is both a well explored area, 
yet one that still receives frequent contributions. However, there is
not satisfactory solution for testing such matchers, which would
require generating positive as well as negative examples for the language. 

We propose an algorithm to generate a language matched by a \emph{generalized}
regular expression with intersection and complement operators.
The complement operator allows us to generate both positive and
negative example words
from a given regular expression.
We implement our generator in \haskell and \ocaml,
and show that its performance are more than
adequate for the purpose of testing.

%%% Local Variables:
%%% mode: latex
%%% TeX-master: "main"
%%% End:

\end{abstract}

\section{Introduction}

Regular languages are everywhere. Due to their apparent simplicity and
their concise representability in the form of regular expressions,
regular languages are used for many text processing
applications, reaching from text editors
\cite{DBLP:journals/cacm/Thompson68} to extracting data from web
pages.

Consequently, there are many algorithms and libraries that implement
parsing for regular expressions. Some of them are based on Thompson's
translation from regular expressions to nondeterministic finite
automata and then apply the powerset construction to obtain a
deterministic automaton. Others are based on Brzozowski's derivatives
\cite{Brzozowski1964} and
map a regular expression directly to a deterministic
automaton. Antimirov's partial derivatives \cite{Antimirov96Partial}
yield another transformation into a nondeterministic automaton. An
implementation based on Glushkov automata has been proposed
\cite{DBLP:conf/icfp/FischerHW10} with decent performance.
Russ Cox's webpage gives a good overview
of efficient implementations of regular expression search. It includes
a discussion of his implementation of Google's RE2 \cite{cox10:_regul_expres_match_wild}.

Some of the algorithms for regular expression matching are rather
intricate and the natural question arises how to test these algorithms. 
While there online repositories with reams of real life regular
expressions \cite{regul_expres_librar}, there are no satisfactory
generators for test inputs. It is not too hard to come up with
generators for strings that match a given regular expression, but that
is only one side of the medal. On the other hand, the algorithm should
reject strings that do not match the regular expression, so it is
equally important to come up with strings that do \textbf{not} match.

This work presents generator algorithms for extended regular expressions that
contain intersection and complement beyond the regular operators. The
presence of the complement operator enables the algorithms to generate
strings that certainly do not match a given (extended) regular
expression.

Our implementations are useful in practice. They are guaranteed to be
productive and produce total outputs. That is, a user can gauge the
string size as well as the number of generated strings without risking
partiality.

Even though the implementations
are not tuned for efficiency they generate
languages at a rate between $1.3\cdot10^3$ and $1.4\cdot10^6$ strings per
second, for Haskell, and up to $3.6\cdot10^6$ strings per second, for
OCaml. The generation rate depends on the density of the language.


\bibliography{../biblio}
\end{document}
