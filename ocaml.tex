\section{\ocaml implementation}
\label{sec:ocaml}

\lstset{language=[Objective]Caml}

In addition to the \haskell implementation, we implemented our language generator
in \ocaml. One of the goal of this implementation was to test
pure \ocaml regular expression libraries such as \mysc{re}\footnote{\url{https://github.com/ocaml/ocaml-re}} and
\mysc{mulet}\footnote{\url{https://github.com/let-def/mulet}}. 
The other was to experiment with strictness and various data-structure for
segments.

Indeed, as long as all the words in each segment are of the same length,
they can be stored in any data structure that support set operations.
In order to test this, we implemented our algorithm as a functor as shown in \autoref{code:sigs}.
Our functor takes as argument two data-structure: words and segments.

\paragraph{Character and Words}

The signature for words in shown in \autoref{code:sigs:word}.
Surprisingly few operations are needed for words: Indeed, we mostly requires the ability to build the empty word (for \code{One}),
words of one char (for \code{Atom}) and to append words
with each others. Notably, we do not require an ordering on words, as we will
see in more details in the description of \code{SEGMENT}.
The \code{length} operation is only required in order to classify
words in segments in the \code{import} function, but is not necessary for the
main algorithm, since all segment operations return words of known lengths.

In practice, such a signature is easily satisfied by the \ocaml \code{string}
type (\ie arrays of bytes), arrays or lists of characters, or by ropes. The
type of individual characters is completely unrestricted.

\paragraph{Segments}



\begin{figure}[h]
  \centering
  \begin{subfigure}{0.44\linewidth}
\begin{lstlisting}[basicstyle=\scriptsize\ttfamily]
module type SEGMENT = sig
  type elt
  type t
  val empty : t
  val is_empty : t -> bool
  val return : elt -> t

  (** Set operations *)
  val append: t -> t -> t
  val union : t -> t -> t
  val inter : t -> t -> t
  val diff : t -> t -> t
  val merge : t OSeq.t -> t

  val of_list : elt list -> t
  val to_seq : t -> elt Sequence.t

  val memoize : t -> t
end
\end{lstlisting}
    \caption{Necessary operation on segments}
    \label{code:sigs:segment}
  \end{subfigure}~
  \begin{subfigure}{0.57\linewidth}
\begin{lstlisting}[basicstyle=\scriptsize\ttfamily]
module type WORD = sig
  type char
  type t
  val empty : t
  val singleton : char -> t
  val length : t -> int
  val append : t -> t -> t
end
\end{lstlisting}
    \caption{Signature for words}
    \label{code:sigs:word}
\begin{lstlisting}[basicstyle=\scriptsize\ttfamily]
module Regenerate
    (Word : WORD)
    (Segment : Segments.S with type elt = Word.t)
: sig
  val gen : 
    sigma:Segment.t -> C.t regex -> Word.t Sequence.t
end
\end{lstlisting}
    \caption{Language generation as a functor}
    \label{code:sigs:regen}
  \end{subfigure}
  \caption{Signatures of the language generator}
  \label{code:sigs}
\end{figure}

%%% Local Variables:
%%% mode: latex
%%% TeX-master: "main"
%%% End:
