\section{\ocaml implementation}
\label{sec:ocaml}

\lstset{language=[Objective]Caml}

In addition to the \haskell implementation, we implemented the language generator
in \ocaml.
The \ocaml version only implements the ``latest'' version of the
algorithm with a segmented representation and fast backward lookup for concatenation and star.
% One of the goal of this implementation was to test
% pure \ocaml regular expression libraries such as ocaml-re\footnote{\url{https://github.com/ocaml/ocaml-re}} and
% mulet\footnote{\url{https://github.com/let-def/mulet}}. 
The main goal of this implementation is to experiment with strictness
and various data structures for segments. 
The key idea is that the internal order on words in a segment does not matter,
because each segment only contains words of the same length.
All we need is a data structure that supports the power series
operations.

To facilitate such experimentation, we implemented the generator
algorithm as a functor whose signature is shown in \autoref{code:sigs}.
The functor takes two data structures as arguments: words and segments.
This structuring allows us to test numerous representations easily without changing
the code. It also forced us to find the ``minimal'' set of operations
needed to implement the algorithm.

\begin{figure}[tp]
  \centering
  \begin{subfigure}{0.44\linewidth}
\begin{lstlisting}[basicstyle=\scriptsize\ttfamily]
module type SEGMENT = sig
  type elt (** Elements *)
  type t (** Segments *)

  val empty : t
  val is_empty : t -> bool
  val return : elt -> t

  (** Set operations *)
  val union : t -> t -> t
  val inter : t -> t -> t
  val diff : t -> t -> t

  (** Product *)
  val append: t -> t -> t

  (** n-way merge *)
  val merge : t list -> t

  (** Import a list *)
  val of_list : elt list -> t

  (** Export elements *)
  val iter : t -> (elt -> unit) -> unit

  (** For transient data-structures *)
  val memoize : t -> t
end
\end{lstlisting}
    \caption{Operations on segments}
    \label{code:sigs:segment}
  \end{subfigure}~
  \begin{subfigure}{0.57\linewidth}
\begin{lstlisting}[basicstyle=\scriptsize\ttfamily]
module type WORD = sig
  type char
  type t
  val empty : t
  val singleton : char -> t
  val append : t -> t -> t
end
\end{lstlisting}
    \caption{Operations on words}
    \label{code:sigs:word}
\begin{lstlisting}[basicstyle=\scriptsize\ttfamily]
module Regenerate
    (Word : WORD)
    (Segment : Segments.S with type elt = Word.t)
: sig
  type lang = Segment.t stream
  val gen : 
    sigma:Segment.t -> C.t regex -> lang
  val iter : lang -> (Word.t -> unit) -> unit
end
\end{lstlisting}
    \caption{Language generation as a functor}
    \label{code:sigs:regen}
  \end{subfigure}
  \caption{Signatures of the language generator}
  \label{code:sigs}
\end{figure}

\paragraph{Character and Words}

\autoref{code:sigs:word} contains the signature for words.
Surprisingly few operations are needed. We need to build the empty word (for \code{One}),
singleton words (for \code{Atom}), and to append two words.
Neither an ordering nor a length operation is needed for words:
Comparison is encapsulated in the segment
data structure and the length of a word is the index of the segment in
which it appears.

This signature is easily satisfied by the \ocaml \code{string}
type (\ie arrays of bytes), arrays, lists of characters, or ropes. The
type of individual characters is unrestricted.

\paragraph{Segments}

\autoref{code:sigs:segment} contains the signature for segments.
The first group of operations creates and tests for empty segments and
singleton segments. 
The main requirement is to support the operations on power series described in
\autoref{sec:ordered-enumeration}.
For the set operations, we need the functions
\code{union}, \code{inter} and \code{inter}.
%
The product described in \autoref{eq:1} is decomposed in two parts:
\begin{itemize}
\item An \code{append} function which implements $U_i V_{n-i}$. It computes the
  product of two segments by pairwise appending their elements.
\item A \code{merge} operation which computes the union of an arbitrary number
  of segments. Its purpose is to collect the segments obtained
  by invocations of \code{append}.
\end{itemize}
%
As the goal is to experiment with different data structures, we want
to use implementations that are transient by default. We thus require 
a function \code{memoize} that avoids recomputing segments accessed
multiple times, which happens when computing the concatenation or the closure of languages. 
%
Finally, the functions  \code{of_list} and \code{iter} import and
export elements to and from a segment.

\subsection{Core algorithm}

The core algorithm is very similar to the Haskell version. The power series
is implemented using a thunk list in the style of \citet{DBLP:conf/cpp/Pottier17}:

\begin{lstlisting}
type 'a node =
  | Nil
  | Cons of 'a * 'a stream
type 'a stream = unit -> 'a node
\end{lstlisting}

A stream is represented by a function which takes a unit argument and returns
a node. A node, in turn, is either \code{Nil} or a \code{Cons} of an
element and the tail of the stream. The empty stream, for instance, is
represented as \code{fun () -> Nil}.
This representation has several advantages\footnote{See
  \url{https://github.com/ocaml/ocaml/pull/1002} for a long discussion
  on the topic.}: it is lazy, fast, lightweight, and almost as easy to
manipulate as regular lists. 
It gives rise to a very natural representation for languages represented as power series:
\begin{lstlisting}[numbers=none]
type lang = Segment.t stream
\end{lstlisting}

The rest of the implementation is similar to the \haskell one. For instance,
here is the implementation of the union of languages:
\begin{lstlisting}
let rec union s1 s2 () = match s1(), s2() with
  | Nil, x | x, Nil -> x
  | Cons (x1, next1), Cons (x2, next2) -> 
    Cons (Segment.union x1 x2, union next1 next2)
\end{lstlisting}
The trailing unit argument, \code{()}, is essential because it allows
us to drive the evaluation of the stream lazily. With this definition, \code{union s1 s2}
will not cause  any evaluation until it is applied to \code{()}.

The concatenation of languages demonstrates how the main algorithm can
be expressed once the segment operations have been abstracted:.
To build $U \cdot V$, we first build the $n$th term
$(U \cdot V)_n = \bigcup_{i=0}^n U_i V_{n-i}$.
We use both \code{Segment.append} to implement the product
of segments and the concatenation of words and \code{Segment.merge} to merge
all the resulting segments.
\begin{lstlisting}
let term_of_length map1 map2 n =
  let combine_segments i =
    Segment.append (IntMap.find i map1) (IntMap.find (n - i) map2)
  in
  List.(range 0 n) |> List.rev_map combine_segments |> Segment.merge
\end{lstlisting}

We then collect all the terms by synchronized recursion over the power series $U$
and $V$:
\begin{lstlisting}
let rec collect n map1 map2 seq1 seq2 () = match seq1 (), seq2 () with
  | Cons (segm1, seq1), Cons (segm2, seq2) ->
    let map1 = IntMap.add n (Segment.memoize segm1) map1 in 
    let map2 = IntMap.add n (Segment.memoize segm2) map2 in
    Cons (term_of_length map1 map2 n, collect (n+1) map1 map2 seq1 seq2)
\end{lstlisting}

The \code{IntMap} module provides a functional implementation of maps
keyed by integers. The maps are used to quickly access
segments of smaller index that have been computed in earlier invocations of
\code{collect}. As such segments are accessed
multiple times, we use \code{memoize} to avoid computing them over and
over again.
Functional maps are sufficient because the size of a map is equal
to the maximum word length, which does not get excessively large.

Finally, we initialize \code{collect} with empty maps.
\begin{lstlisting}[numbers=none]
let concatenate = collect 0 IntMap.empty IntMap.empty
\end{lstlisting}

\subsection{Data Structures}

Now that we implemented the language generator parameterized by segments, we can
experiment with various data structures for segments. We present a
range of potential implementations before comparing their performance.

\subsubsection{Ordered Streams}

The \haskell implementation in \autoref{sec:ordered-enumeration}
represents segments as ordered enumerations. We can use
the same representation in \ocaml thanks to the \code{stream} datatype.
\begin{lstlisting}[numbers=none]
type t = elt stream
\end{lstlisting}

To use an order, we need a comparison and an
\code{append} function on words.  The \code{OrderedMonoid} signature
captures these requirements. \autoref{code:thunklist} shows the
resulting functor \code{ThunkList}.
We omit the straightforward implementation.

\begin{figure}
  \centering
\begin{lstlisting}
module type OrderedMonoid = sig
  type t
  val compare : t -> t -> int
  val append : t -> t -> t
end
module ThunkList (Elt : OrderedMonoid) : SEGMENTS with type elt = Elt.t
\end{lstlisting}
  \caption{Signature for \texttt{ThunkList}}
  \label{code:thunklist}
\end{figure}

The rest of the implementation is not difficult, either. For example, here is the
\code{append} function, where \code{>>=} is bind (or \code{concatMap})
and \code{>|=} is map.

\begin{lstlisting}
let append l1 l2 =
  l1 >>= fun x -> l2 >|= fun y -> Elt.append x y
\end{lstlisting}

The n-way merge is implemented using a priority heap which holds pairs composed
of the head of a stream and its tail. When a new element is required in the
merged stream, we pop the top element of the heap, deconstruct
the tail and insert it back in the heap.

\begin{lstlisting}
let merge l =
  let cmp (v1,_) (v2,_) = K.compare v1 v2 in
  let merge (x1, s1) (_, s2) = (x1, s1@s2) in
  let push h s =
    match s() with Nil -> h | Cons (x, s') -> Heap.insert h (x, [s'])
  in
  let h0 = List.fold_left push (Heap.empty ~cmp ~merge) l in
  let rec next heap () =
    if Heap.is_empty heap then Nil else begin
      let (x, seq), heaps = Heap.pop heap in
      let new_heap = List.fold_left push heaps seq in
      Cons (x, next new_heap)
    end
  in
  next h0
\end{lstlisting}

\subsubsection{Transience and Memoization}

During concatenation and star, we iterate over segments multiple times.
As thunk lists are transient, iterating multiple times over the same list
will compute it multiple times. To avoid this recomputation, we can implement memoization
over thunk lists by pushing the elements in a growing vector as they are
computed:
\begin{lstlisting}
let memoize f =
  let r = CCVector.create () in
  let rec f' i seq () =
    if i < CCVector.length r
    then CCVector.get r i
    else 
      let l = match seq() with
        | Nil -> Nil
        | Cons (x, tail) -> Cons (x, f' (i+1) tail)
      in
      CCVector.push r l;
      l
  in
  f' 0 f
\end{lstlisting}

Such a memoization function incurs a linear cost on streams. To test
if this operation is worthwhile we implemented two modules:
\code{ThunkList} where memoization is the identity and \code{ThunkListMemo}
with the implementation above.

\subsubsection{Lazy Lists}

\ocaml also supports regular lazy lists using the builtin \code{lazy} keyword:

\begin{lstlisting}
type 'a node =
  | Nil
  | Cons of 'a * 'a lazylist
type 'a lazylist = 'a node Lazy.t
\end{lstlisting}

We implemented a \code{LazyList} functor which is identical to the
\code{ThunkList} but uses lazy lists.

\subsubsection{Strict Sets}

As the main operations on segments are set operations, one might 
expect a set implementation to perform well. We implemented segments as sets
of words using \ocaml's built-in \code{Set} module. \ocaml sets are implemented
using binary trees with an imbalance of at most one.
The only operations not implemented by \ocaml's standard library are
the n-way merge and the product, but they are easy to implement with
standard functions such as fold and union.

\subsubsection{Tries}

Tries are prefix trees where each branch is labeled with
a character. A word is considered to belong to the trie if it is possible to
follow its character and reach a value. Tries seem quite amenable to our problem:
\begin{itemize}
\item All words in a segment are of the same length. Thus, they are never prefixes
  of one another and we can use tries where values are only
  at the leaves.
\item The \code{append} operation on tries can be implemented by
  grafting the second trie to all the leaves of the first one.
\end{itemize}

\TODO{}



%%% Local Variables:
%%% mode: latex
%%% TeX-master: "main"
%%% End:
