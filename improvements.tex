\section{Improvements}
\label{sec:improvements}

\subsection{Segmented representation}
\label{sec:segm-repr}

Two operations on the LLO representation, \code{concatenate} and
\code{star}, internally transform their inputs into segment
sequences. Such a transformation forth and back seems wasteful, so we
stipulate to perform the entire generation in terms of segment
sequences. The transformation to a language only happens as the final
step. Thus, the operations in this subsection manipulate languages in
terms of the following type:
\begin{lstlisting}[numbers=false]
type SegLang = [[T.Text]]
\end{lstlisting}

As a second refinement, we try to address the productivity concerns
discussed in Section~\ref{sec:motivation-discussion}. The approach is
to let finite segment sequences represent finite languages. Hence, the
base cases for the interpretation of regular expression may be defined
as in Figure~\ref{fig:base-cases-segmented}.
\begin{figure}[tp]
\begin{lstlisting}
zero :: SegLang
zero = []

one :: SegLang
one = [[T.empty]]

atom :: Char -> SegLang
atom t = [[], [T.singleton t]]
\end{lstlisting}
  \caption{Implementation of base cases for segmented language representation}
  \label{fig:base-cases-segmented}
\end{figure}
The implementation of \code{union}, \code{intersect}, and
\code{difference} (and hence \code{complement}) is an easy exercise and thus omitted.
We discuss the remaining issues with the code for concatenation.
\begin{lstlisting}
concatenate' :: SegLang -> SegLang -> SegLang
concatenate' lx ly = collect 0
  where
    collect n =
      (foldr ILO.union [] $ map (combine n) [0 .. n]) : collect (n+1)
    combine n i =
      [T.append x y | x <- lx !!! i, y <- ly !!! (n - i)]
\end{lstlisting}
It relies on a special indexing operation into a list of lists that returns an empty
list if indexing occurs beyond the end of a segment list:
\begin{lstlisting}
(!!!) :: [[a]] -> Int -> [a]
[]       !!! n = []
(xs:xss) !!! 0 = xs
(xs:xss) !!! n = xss !!! (n - 1)
\end{lstlisting}
This change makes \code{concatenate'} amenable to work with finite
segment lists, 
but any concatenation of languages yields an infinite list of
segments. To avoid that we need a termination criterion for
\code{collect} to determine whether any future call can make a
non-empty contribution.

To this end, we extend the code to keep track of the highest index
seen in \code{lx} and \code{ly}, say, $m_x$ and $m_y$. These indexes
are upper bounds for the length of the longest word in \code{lx} and
\code{ly}, respectively. Hence, if $n > m_x + m_y$, then no word of
length $n$ can be constructed by concatenation of elements from
\code{lx} and
\code{ly}. Figure~\ref{fig:finite-concatenation-segmented} contains
the corresponding extension of \code{concatenate}.
\begin{figure}[tp]
\begin{lstlisting}
concatenate lx ly =
  collect lx ly Nothing Nothing 0
  where
    collect xss yss mmx mmy n =
      let (xss', mmx') = updateMax xss mmx n
          (yss', mmy') = updateMax yss mmy n
          mbound = liftA2 (+) mmx mmy
      in
        case mbound of
          Just m | n > m ->
                   []
          _ ->
            (foldr ILO.union [] $ map (combine n) [0 .. n]) :
            collect xss' yss' mmx' mmy' (n+1)
    combine n i =
      [T.append x y | x <- lx !!! i, y <- ly !!! (n - i)]
    updateMax _ mm@(Just _) n = ([], mm)
    updateMax [] Nothing n = ([], Just n)
    updateMax (_:xss) Nothing n = (xss, Nothing)
\end{lstlisting}
  \caption{Finiteness preserving implementation of concatenation}
  \label{fig:finite-concatenation-segmented}
\end{figure}

The finite cases for \code{star} ($U^* = \{\varepsilon\}$ iff
$U\subseteq\{\varepsilon\}$) are straightforward to detect.

\subsection{Intermediate Assessment}
\label{sec:interm-assessm}




\clearpage{}
%%% Local Variables:
%%% mode: latex
%%% TeX-master: "main"
%%% End:
