\section{Improvements}
\label{sec:improvements}

\subsection{Faster Concatenation by Convolution}
\label{sec:faster-concatenation}
\label{sec:convolution}

Looking back at the code in
Figure~\ref{fig:concatenate-with-segments}, we see that the
invocation of \code{collect n} leads to $n$ invocations of the list
indexing operations \code{lx !! i} and \code{ly !! (n - i)}, which
results in an  overall complexity of $O (n^2)$ for generating the
segment $n$.
% \footnote{We assume that
% the compiler is sufficiently clever to pull \code{ly !!! (n - i)} out of the loop.}

\begin{figure}[tp]
\begin{lstlisting}
concatenate'' :: SegLang -> SegLang -> SegLang
concatenate'' lx ly = collect ly []
  where
    collect (ysegn:ysegs) rly =
      let rly' = ysegn : rly in
      (foldr McIlroy.union []
        $ zipWith (liftA2 T.append) lx rly')
      : collect ysegs rly'
\end{lstlisting}
\vspace{-\baselineskip}
  \caption{Concatenation with convolution}
  \label{fig:concatenation-with-convolution}
\end{figure}
The \code{combine} operation is an example of a convolution where
indexing of the two lists proceeds in opposite directions. The \code{collect} operation
could take advantage of this pattern and build a reversed list \code{rly} of already
processed segments of \code{ly} at $O(1)$ extra cost. Using this additional data structure, the
convolution can be implemented in linear time using
\code{zipWith} as shown in Figure~\ref{fig:concatenation-with-convolution}.

The code in the figure only handles infinite lists of
segments. When extended with handling the case for finite lists,
it gives rise to an optimization. If \code{lx} is finite,
then the \code{zipWith} automatically stops processing unnecessary indices in
\code{ly}. Conversely, if \code{ly} starts with some empty segments
because all its words are longer than some lower bound, then these
segments could safely be omitted from \code{rly}.

This consideration reveals that the algorithm implemented in
\code{concatenate''} is inherently asymmetric because it does not
exploit the finiteness of \code{ly}. In the finiteness-aware version, this asymmetry can be
addressed easily.  The code can be
enhanced to detect that \code{ly} is finite. Then it just flips the roles of
\code{lx} and \code{ly}, so that from now on \code{ly} is traversed in 
forward direction and \code{lx} backwards. Flipping roles at length $n$ requires taking
the initial $(n+1)$ segments of \code{lx} and reversing the resulting
list. The cost for this reversal is amortized by the previous traversal of
\code{ly} up to this point, so it has no impact on the asymptotic
complexity.

We implemented the finiteness-aware algorithm including role flipping,
but we do not include the code in the paper, as it is a
tedious variation of the code in
Figure~\ref{fig:concatenation-with-convolution}. 

% \subsubsection{Map Data Structure}\label{sec:map-data-structure}
% As segments represent finite sets, we could store the segments of \code{lx} and \code{ly}
% up to length \code{n} in a map data structure of type
% \code{Data.Map.Map Int [T.Text]}, which maps the number $i$ to the
% segment that contains the words of length $i$. This change would bring
% down the quadratic complexity to $O (n\log n)$. 

% \subsubsection{Sparse Indexing}\label{sec:sparse-indexing}
% Along with the recursive calls of \code{collect}, we could accumulate
% a list/set of indexes for \code{lx} and \code{ly} such that the
% corresponding segments are non-empty. With this information, the
% \code{combine} operation could be targeted to only consider the useful
% indexes where both \code{lx !!! i} and \code{ly !!! (n - i)} are
% non-empty.

% The combination of using maps and sparse indexing gives rise to a
% more symmetric behavior.


\subsection{Refined Segment Representation}
\label{sec:segm-repr}

The refined segment representation attempts to represent finite
languages by finite segment sequences as much as possible. Hence, the
base cases for the interpretation of a regular expression may be
defined as in Figure~\ref{fig:base-cases-segmented}.
\begin{figure}[tp]
\begin{lstlisting}
zero :: SegLang
zero = []
one :: SegLang
one = [[T.empty]]
atom :: Char -> SegLang
atom t = [[], [T.singleton t]]
\end{lstlisting}
  \caption{Base cases for refined segment representation}
  \label{fig:base-cases-segmented}
\end{figure}
The implementation of \code{union}, \code{intersect}, 
\code{difference}, and \code{complement} is an easy exercise and thus omitted.
We discuss the remaining issues with the previous code for concatenation in
Figure~\ref{fig:concatenation-with-convolution}. 
% The list indexing \lstinline{!!} in that code relies on languages being infinite lists. 
% It must be replaced by a special indexing operation into a list of lists that returns an empty
% list if indexing occurs beyond the end of a segment list:
% \begin{lstlisting}
% (!!!) :: [[a]] -> Int -> [a]
% []       !!! n = []
% (xs:xss) !!! 0 = xs
% (xs:xss) !!! n = xss !!! (n - 1)
% \end{lstlisting}
% Changing \lstinline{!!} to \lstinline{!!!} makes \code{concatenate} amenable to work with finite
% segment lists. 

Function \lstinline{concatenate''} always returns an infinite list of
segments. To keep that list finite, we need
a termination criterion for \code{collect} to determine that all
future calls will only contribute empty segments.

To this end, we keep track of the lowest index
seen in \code{lx} and \code{ly} where the respective segment list is
exhausted, say, $m_x$ and $m_y$ (if they exist). These indexes
are upper bounds for the length of the longest word in \code{lx} and
\code{ly} such that $\forall x \in \code{lx}$ it must be that
$|x|\le m_x-1$ and analogously for \code{ly}. Then we apply the
following lemma.
\begin{lemma}
  Let $X, Y \subseteq \Sigma^*$ be languages and suppose that there
  are numbers $m_x, m_y \ge 0$ such that
  \begin{align*}
    & \forall x \in X, |x| < m_x
    && \forall y \in Y, |y| < m_y
  \end{align*}
  Then $\forall w\in X\cdot Y$, $|w| < m_x + m_y -1$.
\end{lemma}
To prove the lemma, observe that the longest word in
the product is bounded by $|xy| = |x| + |y| \le m_x -1 + m_y -1$. In
consequence, if $n \ge m_x + m_y - 1$, then no word of
length $n$ can be constructed by concatenation of elements from
\code{lx} and
\code{ly}.
We omit the straightforward modification of the code.
% Figure~\ref{fig:finite-concatenation-segmented} contains
% the corresponding extension of \code{concatenate}. It makes use of
% \code{liftA2} used at type
% \code{(Int -> Int -> Int) -> Maybe Int -> Maybe Int -> Maybe Int}
% to add two values of type \code{Maybe
%   Int} if they are present.\footnote{The function \lstinline{liftA2 ::
%     Applicative f => (a -> b -> c) -> f a -> f b -> f c} is taken from
% the standard module \lstinline{Control.Applicative}.}
% \begin{figure}[tp]
% \begin{lstlisting}
% concatenate lx ly =
%   collect lx ly Nothing Nothing 0
%   where
%     collect xss yss mmx mmy n =
%       let (xss', mmx') = updateMax xss mmx n
%           (yss', mmy') = updateMax yss mmy n
%           mbound = liftA2 (+) mmx mmy
%       in
%         case mbound of
%           Just m | n >= m - 1 ->
%                    []
%           _ ->
%             (foldr ILO.union [] $ map (combine n) [0 .. n]) :
%             collect xss' yss' mmx' mmy' (n+1)
%     combine n i =
%       [T.append x y | x <- lx !!! i, y <- ly !!! (n - i)]
%     updateMax _ mm@(Just _) n = ([], mm)
%     updateMax [] Nothing n = ([], Just n)
%     updateMax (_:xss) Nothing n = (xss, Nothing)
% \end{lstlisting}
%   \caption{Finiteness preserving implementation of concatenation}
%   \label{fig:finite-concatenation-segmented}
% \end{figure}

The output of \code{star} is only finite in two cases as $U^* = \{\varepsilon\}$ iff
$U\subseteq\{\varepsilon\}$. Otherwise $|U^*|$ is infinite.
The finite cases are straightforward to detect and thus not further discussed.



Segmentation yields a finite representation for all finite languages
defined without using the \code{complement} operation. While
$\Lang{\Rcomplement{(\Rcomplement{\Rnull})}} = \emptyset$ is a finite 
language, the output of \lstinline{complement (complement zero)} is still an
infinite list of empty segments.

\S~\ref{sec:more-finite-repr}  discusses ways to get finite
representations from more languages. Generally it is
impossible to guarantee a finite language is represented by a
finite stream of segments.


\subsection{Faster Closure}
\label{sec:faster-closure}

The optimizations described for concatenation also apply to the
computation of the Kleene closure. The convolution approach does not
require any flipping in this case because it is clear that only the input language
can be finite, as the cases where the output language is finite are
treated separately.  

\subsection{More Finite Representations}
\label{sec:more-finite-repr}

We already remarked that we can keep segmented representations finite
for finite languages constructed without using \code{complement}.  To
extend the realm of finite representations we propose to use a custom
data types for segment lists.
\begin{center}
  \vspace{-\baselineskip}
  \begin{minipage}[t]{0.49\linewidth}
\begin{lstlisting}
data Segments
  = Empty
  | Cons Lang Segments
  | Full [Lang]
\end{lstlisting}
  \end{minipage}
  \qquad
  \begin{minipage}[t]{0.4\linewidth}
\begin{lstlisting}
data Lang
  = Null
  | Data [T.Text]
  | Univ [T.Text]
\end{lstlisting}
  \end{minipage}
\end{center}
Constructor \code{Empty} represents the empty set. A segment list
\code{Cons xl xsegs} represents the union of the language
\code{xl} and segments \code{xsegs}. If a \code{Cons} node
appears at level $n\ge0$ in a \code{Segments} data structure, then all
words in \code{xl} have length $n$. The constructor \code{Full xls} is
the novelty of this type. If it appears at level $n$, then it
represents all words in $\Sigma^*$ of length $\ge n$. For convenience,
the argument \code{xls} contains these words, structured as a
(standard) list of segments.

The definition of \code{Segments} relies on data type \code{Lang} to
represent languages $L\subseteq \Sigma^n$,  for some $n$.
Constructor \code{Null} stands for the empty set, \code{Data ws}
stands for a non-empty set of words represented by an increasingly
ordered list \code{ws}, and \code{Univ ws}, when encountered at level
$n$, indicates that its argument \code{ws} represents the full set $\Sigma^n$.

It is an easy exercise to implement the operations
\code{union}, \code{intersect}, and \code{difference} on \code{Lang}
and \code{Segments} in a way that preserves the above invariants as
much as possible.

The resulting generation algorithm solves our previous problem 
with $\Lang{\Rcomplement{(\Rcomplement{\Rnull})}} = \emptyset$,
because it evaluates to \code{Empty}. Also, $\Lang{\Rcomplement{a}}$
evaluates to a finite representation:\footnote{We use string notation
  for elements of \lstinline{Data.Text.Text} for readability. The bar in \lstinline{Full _} stands for an unevaluated infinite list of full segments.}
\begin{lstlisting}[numbers=none]
Cons (Univ [""]) (Cons (Data ["b"]) (Full _))
\end{lstlisting}
But a slight variation like
$\Lang{(\Rcomplement{a})(\Rcomplement{b})} = \{a,b\}^*$ would not be
represented finitely.

We can extend the range of languages with
finite representations by dualizing the idea for
detecting finiteness in \S~\ref{sec:segm-repr}. The following
lemma captures this idea.
\begin{lemma}
Let $X, Y \subseteq \Sigma^*$ be languages and 
suppose that there are numbers $f_x, f_y\ge0$ such that
\begin{align*}
  & \forall x \in \Sigma^*,  |x| \ge f_x \Rightarrow x \in X
  && \forall y \in \Sigma^*, |y| \ge f_y \Rightarrow y \in Y
\end{align*}
Then $\forall w \in \Sigma^*$, $|w|\ge f_x+f_y \Rightarrow w \in X\cdot Y$.
\end{lemma}
The generation algorithm can determine $f_x$
and $f_y$ by detecting when a segment list is \code{Full}. Once $f_x$
and $f_y$ are both determined and the generated word length $n$ is
greater than or equal to $f_x+f_y$, then we can finish producing the segments by
outputting the appropriate \code{Full}.

This approach yields finite segment representations
for many finite and co-finite languages. % For example, the computed
% representations for $\Rcomplement a$, $\Rcomplement b$, and
% ${(\Rcomplement{a})(\Rcomplement{b})}$ are as follows. 
% \begin{lstlisting}[numbers=none]
% Cons (Univ [""]) (Cons (Data ["b"]) (Full _))
% Cons (Univ [""]) (Cons (Data ["a"]) (Full _))
% Cons (Univ [""]) (Cons (Data ["a", "b"]) (Full _))
% \end{lstlisting}
But the algorithm is easy to defeat. For example, both
$\Lang{\Runion{\Rstar a} \Rcomplement{(\Rstar a)}} = \Sigma^*$ and
$\Lang{\Rintersect{\Rstar a} \Rcomplement{(\Rstar a)}} = \emptyset$
are both mapped to infinite segment lists.

%%% Local Variables:
%%% mode: latex
%%% TeX-master: "main"
%%% End:
