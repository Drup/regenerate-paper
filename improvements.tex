\section{Improvements}
\label{sec:improvements}

\subsection{Segmented representation}
\label{sec:segm-repr}

Two operations on the LLO representation, \code{concatenate} and
\code{star}, internally transform their inputs into segment
sequences. Such a transformation forth and back seems wasteful, so we
stipulate to perform the entire generation process in terms of segment
sequences. The transformation to a language only happens as the final
step. Thus, the operations in this subsection manipulate languages in
terms of the following type:
\begin{lstlisting}[numbers=none]
type SegLang = [[T.Text]]
\end{lstlisting}

As a second refinement, we try to address the productivity concerns
discussed in Section~\ref{sec:motivation-discussion}. The approach is
to let finite segment sequences represent finite languages. Hence, the
base cases for the interpretation of regular expression may be defined
as in Figure~\ref{fig:base-cases-segmented}.
\begin{figure}[tp]
\begin{lstlisting}
zero :: SegLang
zero = []

one :: SegLang
one = [[T.empty]]

atom :: Char -> SegLang
atom t = [[], [T.singleton t]]
\end{lstlisting}
  \caption{Implementation of base cases for segmented language representation}
  \label{fig:base-cases-segmented}
\end{figure}
The implementation of \code{union}, \code{intersect}, and
\code{difference} (and hence \code{complement}) is an easy exercise and thus omitted.
We discuss the remaining issues with the preliminary code for concatenation in
Figure~\ref{fig:preliminary-concatenation-segmented}. 
\begin{figure}[tp]
\begin{lstlisting}
concatenate' :: SegLang -> SegLang -> SegLang
concatenate' lx ly = collect 0
  where
    collect n =
      (foldr ILO.union [] $ map (combine n) [0 .. n]) : collect (n+1)
    combine n i =
      [T.append x y | x <- lx !!! i, y <- ly !!! (n - i)]
\end{lstlisting}
  \caption{Preliminary implementation of concatenation for segmented
    language representation}
  \label{fig:preliminary-concatenation-segmented}
\end{figure}
It relies on a special indexing operation into a list of lists that returns an empty
list if indexing occurs beyond the end of a segment list:
\begin{lstlisting}
(!!!) :: [[a]] -> Int -> [a]
[]       !!! n = []
(xs:xss) !!! 0 = xs
(xs:xss) !!! n = xss !!! (n - 1)
\end{lstlisting}
This change makes \code{concatenate'} amenable to work with finite
segment lists, 
but any concatenation of languages yields an infinite list of
segments. To avoid that we need a termination criterion for
\code{collect} to determine whether any future call can make a
non-empty contribution.

To this end, we extend the code to keep track of the lowest index
seen in \code{lx} and \code{ly} where the respective segment list is
exhausted, say, $m_x$ and $m_y$ (if they exist). These indexes
are upper bounds for the length of the longest word in \code{lx} and
\code{ly} such that $\forall x \in \code{lx}$ it must be that
$|x|\le m_x-1$ and analogously for \code{ly}. Then we apply the
following lemma.
\begin{lemma}
  Let $X, Y \subseteq \Sigma^*$ be languages and suppose that there
  are numbers $m_x, m_y \ge 0$ such that
  \begin{align*}
    & \forall x \in X, |x| < m_x
    && \forall y \in Y, |y| < m_y
  \end{align*}
  Then $\forall w\in X\cdot Y$, $|w| < m_x + m_y -1$.
\end{lemma}
To prove the lemma, observe that the longest word in
the product is bounded by $|xy| = |x| + |y| \le m_x -1 + m_y -1$. In
consequence, if $n \ge m_x + m_y - 1$, then no word of
length $n$ can be constructed by concatenation of elements from
\code{lx} and
\code{ly}. Figure~\ref{fig:finite-concatenation-segmented} contains
the corresponding extension of \code{concatenate}. It makes use of
\code{liftA2} used at type \code{(Int -> Int -> Int) -> Maybe Int ->
  Maybe Int -> Maybe Int} to add two values of type \code{Maybe
  Int} if they are present.\footnote{The function \lstinline{liftA2 ::
    Applicative f => (a -> b -> c) -> f a -> f b -> f c} is taken from
the standard module \lstinline{Control.Applicative}.}
\begin{figure}[tp]
\begin{lstlisting}
concatenate lx ly =
  collect lx ly Nothing Nothing 0
  where
    collect xss yss mmx mmy n =
      let (xss', mmx') = updateMax xss mmx n
          (yss', mmy') = updateMax yss mmy n
          mbound = liftA2 (+) mmx mmy
      in
        case mbound of
          Just m | n >= m - 1 ->
                   []
          _ ->
            (foldr ILO.union [] $ map (combine n) [0 .. n]) :
            collect xss' yss' mmx' mmy' (n+1)
    combine n i =
      [T.append x y | x <- lx !!! i, y <- ly !!! (n - i)]
    updateMax _ mm@(Just _) n = ([], mm)
    updateMax [] Nothing n = ([], Just n)
    updateMax (_:xss) Nothing n = (xss, Nothing)
\end{lstlisting}
  \caption{Finiteness preserving implementation of concatenation}
  \label{fig:finite-concatenation-segmented}
\end{figure}

The output of \code{star} is only finite in two cases as $U^* = \{\varepsilon\}$ iff
$U\subseteq\{\varepsilon\}$. Otherwise $|U^*|$ is infinite.
The finite cases are straightforward to detect and thus not further discussed.

\subsection{Assessment}
\label{sec:interm-assessm}

The segmented representation has a number of features that makes it
well suited in practice. We discuss these features in turn and
consider further minor improvements in the subsections to follow.

\subsubsection{Productivity}
Each generalized regular expression is interpreted by a productive
(finite or infinite) stream of segments. Each segment is guaranteed to
be a total, finite list of words. There is no partiality in this representation.

\subsubsection{Easy Bounding}
To restrict the generated segmented output \code{segs} to words of length less than a given
bound \code{n}, all we need to do is \code{concat (take n segs)}. The result is
a finite list of words. In contract, such filtering is not effective
for the LLO representation where
\lstinline{takeWhile (\w -> T.length w <  n) llo} may yield a partial list.

\subsubsection{Finite Languages}
Segmentation yields a finite representation for all finite languages
defined without using the \code{complement} operation. While
$\Lang{\Rcomplement{(\Rcomplement{\Rnull})}} = \emptyset$ is a finite 
language, the output of \code{complement (complement zero)} is an
infinite list of empty segments.

In Section~\ref{sec:more-finite-repr}, we discuss ways to get finite
representations from more languages, but in general it is
not possible to guarantee that a finite language is represented by a
finite stream of segments.

\subsection{Faster Concatenation}
\label{sec:faster-concatenation}

Looking back at the code in
Figure~\ref{fig:preliminary-concatenation-segmented}, we see that the
invocation of \code{collect n} leads to $n$ invocations of the list
indexing operations \code{lx !!! i} and \code{ly !!! (n - i)}, which
results in an  overall complexity of $O (n^2)$ for generating the
segment $n$.\footnote{We assume that
GHC is sufficiently clever to pull \code{ly !!! (n - i)} out of the loop.}

There are several different ways to reduce this complexity.

\subsubsection{Convolution}
\label{sec:convolution}

\begin{figure}[tp]
\begin{lstlisting}
concatenate'' :: SegLang -> SegLang -> SegLang
concatenate'' lx ly = collect ly []
  where
    collect (ysegn:ysegs) rly =
      let rly' = ysegn : rly in
      (foldr ILO.union [] $ zipWith (liftA2 T.append) lx rly') :
      collect ysegs rly'
\end{lstlisting}
  \caption{Concatenation with convolution}
  \label{fig:concatenation-with-convolution}
\end{figure}
The \code{combine} operation is an example of a convolution where
indexing of the two lists proceeds in opposite directions. The \code{collect} operation
could take advantage of this pattern and build a reversed list \code{rly} of already
processed segments of \code{ly} at $O(1)$ extra cost. Using this additional data structure, the
convolution can be implemented in linear time using
\code{zipWith} as shown in Figure~\ref{fig:concatenation-with-convolution}.

For simplicity, the code in the figure only handles infinite lists of
segments. But when extended with handling the case for finite lists,
it gives rise to a very nice optimization. If \code{lx} is finite,
then the \code{zipWith} automatically stops processing unnecessary indices in
\code{ly}. Conversely, if \code{ly} starts with some empty segments
because all its words are longer than some lower bound, then these
segments could safely be omitted from \code{rly}.

This consideration reveals that the algorithm implemented in
\code{concatenate''} is inherently asymmetric because it does not
exploit the finiteness of \code{ly}. In the finiteness-aware version, this asymmetry can be
addressed easily.  The code can be
enhanced to detect that \code{ly} is finite. Then it just flips the roles of
\code{lx} and \code{ly}, so that from now on \code{ly} is traversed in 
forward direction and \code{lx} backwards. Flipping roles at length $n$ requires taking
the initial $(n+1)$ segments of \code{lx} and reversing the resulting
list. The cost for this reversal is amortized by the previous traversal of
\code{ly} up to this point, so it has no impact on the asymptotic
complexity.

We implemented the finiteness-aware algorithm including role flipping,
but we do not include the code in the paper, as it is a
straightforward, but tedious variation of the code in
Figure~\ref{fig:concatenation-with-convolution}. 

\subsubsection{Map data structure}
Alternatively, we could store the segments of \code{lx} and \code{ly}
up to length \code{n} in a map data structure of type
\code{Data.Map.Map Int [T.Text]}, which maps the number $i$ to the
segment that contains the words of length $i$. This change would bring
down the quadratic complexity to $O (n\log n)$. 

\subsubsection{Sparse indexing}
Along with the recursive calls of \code{collect}, we could accumulate
a list/set of indexes for \code{lx} and \code{ly} such that the
corresponding segments are non-empty. With this information, the
\code{combine} operation could be targeted to only consider the useful
indexes where both \code{lx !!! i} and \code{ly !!! (n - i)} are
non-empty.

The combination of using maps and sparse indexing gives rise to a
more symmetric behavior.

\subsection{Faster Closure}
\label{sec:faster-closure}

The optimizations described for concatenation also apply to the
computation of the Kleene closure. The convolution approach does not
require any flipping because it is clear that only the input language
can be finite, as the cases where the output language is finite are
treated specially, anyway.  

Using a map data structure and sparse indexing can be done in exactly
the same way as for concatenation.

\subsection{More Finite Representations}
\label{sec:more-finite-repr}

We already remarked that we can keep segmented representations finite
for finite languages constructed without using \code{complement}. An
obvious idea to extend the realm of finite representations is to use a
custom data type \code{Segments} for segment lists.
\begin{lstlisting}
data Segments
  = Empty
  | Cons Lang Segments
  | Full [Lang]
\end{lstlisting}
The constructor \code{Empty} represents the empty set. A segment list
of the form \code{Cons xl xsegs} represents the union of the language
\code{xl} and the segments \code{xsegs}. If such a \code{Cons} node
appears at level $n\ge0$ in a \code{Segments} data structure, then all
words in \code{xl} have length $n$. The constructor \code{Full xls} is
the novelty of this type. If it appears at level $n$, then it
represents all words in $\Sigma^*$ of length $\ge n$. For convenience,
the argument \code{xls} contains these words, structured as a
(standard) list of segments.

The definition of \code{Segments} relies on a special data type \code{Lang} to
represent languages $L\subseteq \Sigma^n$,  for some $n$.
\begin{lstlisting}
data Lang
  = Null
  | Data [T.Text]
  | Univ [T.Text]
\end{lstlisting}
The constructor \code{Null} stands for the empty set, \code{Data ws}
stands for a non-empty set of words represented by an increasingly
ordered list \code{ws}, and \code{Univ ws}, when encountered at level
$n$, indicates that its argument \code{ws} represents the full set of
word $\Sigma^n$.

It is an easy, but tedious exercise to implement the operations
\code{union}, \code{intersect}, and \code{difference} on \code{Lang}
and \code{Segments} in a way that preserves the above invariants as
much as possible.

The resulting generation algorithm solves our previous problem 
with $\Lang{\Rcomplement{(\Rcomplement{\Rnull})}} = \emptyset$,
because it evaluates to \code{Empty}. Also, $\Lang{\Rcomplement{a}}$
evaluates to a finite representation:\footnote{We use string notation
  for elements of \lstinline{Data.Text.Text} for readability.}
\begin{lstlisting}[numbers=none]
Cons (Univ [""]) (Cons (Data ["b"]) (Full _))
\end{lstlisting}
But a slight variation like
$\Lang{(\Rcomplement{a})(\Rcomplement{b})} = \{a,b\}^*$ would not be
represented finitely.

It turns out that we can extend the range of languages for which
finite representations are generated by dualizing the idea for
detecting finiteness in Section~\ref{sec:segm-repr}. The following
lemma captures this idea.
\begin{lemma}
Let $X, Y \subseteq \Sigma^*$ be languages and 
suppose that there are numbers $f_x, f_y\ge0$ such that
\begin{align*}
  & \forall x \in \Sigma^*,  |x| \ge f_x \Rightarrow x \in X
  && \forall y \in \Sigma^*, |y| \ge f_y \Rightarrow y \in Y
\end{align*}
Then $\forall w \in \Sigma^*$, $|w|\ge f_x+f_y \Rightarrow w \in X\cdot Y$.
\end{lemma}
 An algorithm analogous 
to the finiteness-preserving one in
Figure~\ref{fig:finite-concatenation-segmented} can determine $f_x$
and $f_y$ by detecting when a segment list is \code{Full}. Once $f_x$
and $f_y$ are both determined and the generated word length $n$ is
greater than or equal to $f_x+f_y$, then we can finish the segments by
outputting the appropriate \code{Full}.

This algorithm generates finite segmented representations
for many finite and co-finite languages. For example, the computed
representations for $\Rcomplement a$, $\Rcomplement b$, and
${(\Rcomplement{a})(\Rcomplement{b})}$ are as follows. 
\begin{lstlisting}[numbers=none]
Cons (Univ [""]) (Cons (Data ["b"]) (Full _))
Cons (Univ [""]) (Cons (Data ["a"]) (Full _))
Cons (Univ [""]) (Cons (Data ["a", "b"]) (Full _))
\end{lstlisting}
But the algorithm is easy to defeat. For example, consider
$\Lang{\Runion{\Rstar a} \Rcomplement{(\Rstar a)}} = \Sigma^*$ or
$\Lang{\Rintersect{\Rstar a} \Rcomplement{(\Rstar a)}} = \emptyset$,
which are both mapped to infinite segment lists.

%%% Local Variables:
%%% mode: latex
%%% TeX-master: "main"
%%% End:
