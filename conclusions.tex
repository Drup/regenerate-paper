\section{Conclusions and Future Work}
\label{sec:conclusions}

In this article, we presented an algorithm to generate the language
of a \emph{generalized} regular expression with union, intersection
and complement operators. Using this technique, we can generate both
positive and negative instance of a regular expression, thus enabling
easier testing of regular expression parsers without an oracle.
We provide two implementations: one in \haskell which explores
different algorithmic improvements and one in \ocaml which evaluates choices
in data structures. We then measured the performance of these implementations.

Even though our implementations are not heavily optimized, our approach generates
languages at a rate that is more than sufficient for testing
purposes, between $1.3\cdot10^3$ and $1.4\cdot10^6$ strings per seconds.
We can then combine our generator with property based testing
to test regular expression parsers on randomly-generated regular expressions.
While our approach eliminated the need for an oracle, the burden of correctness
now lies on the language generator. We would like to implement our algorithm
in Agda and prove its correctness and its productivity.

We also want to extend our approach to a more general context.
Notably, we can consider new regular expression operators.
While some operators are very easy to implement, such as option and generalized repetition,
other would cause some additional challenge, such as lookaround and boundaries
operators. We can also consider non-regular operators! Indeed, our approach
is compositional, and can thus scale to decidable languages.
It remains to be seen if context free languages can also be generated
with this method.
Representing regular languages as formal power series provides numerous avenues
for extensions. In this article, we used the semiring
of languages of size $n$. Other semirings might also yield interesting
algorithms. For instance, using the stream of boolean coefficients would
allow to generate the language and its complement in one go. We could
also use a richer semiring to generate \emph{weighted} words.

Finally, in \cref{sec:bench}, we saw that the performance of our
generator highly depends
both on the shape of the language and the shape of the regular expression.
We noticed a rough correlation with the size of each segments, but did not
provide a more precise account. One might wonder if such an account is possible
and if yes, can we use this information to improve language generation
by transforming a regular expression to an equivalent yet more efficient one.

%%% Local Variables:
%%% mode: latex
%%% TeX-master: "main"
%%% End:
